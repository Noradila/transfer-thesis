\chapter{Multichannel Routing Protocol (Prot and Algo)}
\label{MCRP}

In this chapter, we focus specifically on the reliability of radio communication links in sensor networks. WSNs often suffer from frequent occurrences of external interference such as Wi-Fi and Bluetooth. Multi-channel feature (solution) is use to mitigate external external interference to ensure a seamless operation in the case of severe external interference.
Multichannel communication in wireless networks can alleviate the effects of interference to enable WSNs to operate reliably in the presence of such interference. As a result can improve the network efficiency of spectrum usage, network stability, link reliability and minimize latency and minimal number of packet loss, hence, retransmission.

As a multichannel solution, we present MCRP, a decentralized cross-layer protocol with a centralized controller. Our cross layer multi-channel protocol focuses on the network and application layers. This allows channel assignment decisions to be made thoroughly without being limited by the low layer complexity. The system has two parts: a central algorithm which is typically run by the LPBR and selects which channel each node should listen on; and a protocol which allows the network to communicate the channel change decision, probe the new channel and either communicate the success of the change or fall back to the previous channel. MCRP concentrates on finding channels for the nodes that are free from or have low interference. It allows the allocation of these channels in a way likely to minimize the chances of nodes which are physically near to communicate on the same channel. Hence, it reduces cross interference between different pairs of nodes.

\section{MCRP Design}
Before presenting the design on MCRP and the main components, we introduce the general design goals (several crucial observations). The design of the multichannel protocol is motivated/based on several crucial observations: 

\begin{enumerate}[i. ]
\item Channel assignment - Sensors have limited memory and battery capabilities. In order to maximize the sensors lifetime, a centralized LPBR that has larger memory and fully powered in used for decision making. LPBR has complete knowledge of the topology which enables it to make good channel assignment decisions based on a two-hop colouring algorithm – centralized; thus nodes computation is transferred to LPBR.

\item Interference - External interference cannot be predicted, thus channels cannot be allocated beforehand as it varies over time and locations. It is impossible to determine a single channel that is free from interference at any location. Our protocol checks the channel condition each time before deciding on a channel change to reduce interference and maximize throughput. //Transmission collisions may occur in wireless networks, especially with bursty traffic that may be present in sensor network. //Interference may lead to packet losses which need to be catered for with suitable retransmission mechanisms.

\item Frequency diversity - Multichannel increases the robustness of the network towards interference. However, applying multichannel to the existing RPL may hinder detection of the new nodes and cause problems for maintaining the RPL topology. We overcome thus problem by two mechanisms. (((((??)Existing nodes maintain a table of channels on which their neighbours listen and use unicast to contact those nodes. New nodes listen on a Contiki default channel (26) and when connecting search through all channels. As in RPL, periodically all nodes broadcast RPL control messages on the default channel in an attempt to contact new nodes.
\end{enumerate}

Our work (make use) of existing standards and focus on improvement that can be used with the standards. MCRP is compatible with RPL with minor changes in order to be able to be used as a multichannel protocol as MCP concentrates on cross layers between the network and application layers. Minor changes on the MAC layer (ContikiMAC – that is energy efficient DETAILS???) in order to be compatible with multichannels – to be able to change to the correct channel when transmitting/retransmitting by accessing the channel information that are stored on the network layer.

\section{Channel Selection Strategy}
One main advantage of the system we propose is generality. Any algorithm can be used at the LPBR to assign channels. In this paper we use a two-hop colouring algorithm to select a channel to be assigned to a node.
The two-hop colouring algorithm attempts to ensure that nearby nodes do not communicate on the same channel and risk interfering with each other. The protocol is inspired by the graph colouring problems \cite{graphColouring}. The core idea is that no node should use the same listening channel as a neighbour or a neighbour of a neighbour (two hops).
This allows fair load balancing on the channels and reduces channel interference that could occur when two nearby nodes transmit together on the same channel. The nodes used in this paper have a transmission range of approximately 20-30 metres indoors and 75-100 metres outdoors \cite{telosb-datasheet}. It could be the case that many nodes in a sensor network are in the transmission range of each other and potentially interfered with.
 
All nodes are initialised to channel 26 which is the common default channel for Contiki MAC layer since it often has fewer interference problems with Wi-Fi and other sources. The studies in \cite{chrysso, micmac, watteyne} use a set list of whitelisted channels in their experiments and have channel 26 in common. The usual RPL set up mechanism is used to exchange control messages that are required to form an optimised topology before channel assignments can take place. The nodes will only be on the same channel once during the initial setup.
%new add
This enables the node to detect and find nearby neighbours that are in range before it can decides on the best route based on the list of neighbours it can be connected to. 
	
In the two-hop colouring algorithm, the LPBR chooses a node to which it will assign a channel to listen on. The selection is random (from channels 11 to 26) based on the full range available \cite{ieee802.15.4}. 

The value is random - no need to pseudo-random number generators to avoid node persistently generating the same numbers as the channel that is bad for one node might gives good result for another node depending on the location of the node which might not be within where the channel is bad at. 

The protocol checks neighbours and neighbours of neighbours to see if any of those are listening on this channel already. If any are, a new channel is picked from the remaining list of available channels. If the LPBR has knowledge of existing bad channels then those channels can be blacklisted.  Knowledge of channel interference which is gained by probing can be used to decide that a channel should not be used. If a channel is found then the channel switching protocol is triggered. If no channel can be found meeting these conditions, the current channel is kept.  

The node selection algorithm must only attempt one channel change at a time to ensure probing is done on the correct new channel and for the node to finalise the channel to be used before another node attempts a channel change.
The protocol ascertains that the channel change attempt will always result in a message returned to the LPBR either confirming the new channel or announcing a reversion to the old channel. Until one or other of these happens, no new channel change will be made to enable the neighbours transmitting on the correct channel.

\section{Channel Switching}


\begin{figure}
\centering
\includegraphics[trim=2cm 2cm 3cm 2cm, clip=true, totalheight=0.6\textheight, angle=270]{channelSwitching.pdf}
\caption{Channel switching processes}
\label{fig_mcrp}
\end{figure}

Figure \ref{fig_mcrp} shows the state machine for the channel switching protocol.
As explained in the previous section, a choice of a new channel by the channel selection protocol causes a change channel message to be sent to the appropriate node. 
Upon receiving a channel change message, a node $N$ stores its current channel $C$ and communicates to all its neighbours the new channel $D$ that it wishes to change to. Those neighbours will update their neighbour tables to ensure that they now send to node $N$ on channel $D$.  The node $N$ begins the channel quality checking process with each neighbour in turn by sending them a probe request. If this process fails for any neighbour then the node reverts to channel $C$. If all channel quality checks succeed, the node $N$ will listens on channel $D$. In both cases, node $N$ informs its neighbours of the decision to channel $C$ or $D$ and informs the LPBR of the channel checking results. The channel checking process uses probe packets that might interfere with other transmissions temporarily. However, it is important to emphasise that the network remains fully functional and connected at all stages of this protocol.

\section{Channel Quality Checking}
The channel quality checking is invoked each time a node changes channel after receiving a message from the LPBR. A node $N$ changing to channel $D$ informs all neighbours in turn, of the new channel $D$ it will be listening on as described in the previous section. It then enters the \emph{Probe Wait} state and begins channel quality checking with each tree neighbour in turn. In describing the channel quality checking process, it is worth emphasising the  distinction between neighbours and tree neighbours. Node neighbours are all nodes that a given node knows it could transmit to. Tree neighbours are the nodes that a node does transmit to through the topology formed by the RPL protocol. 

In the \emph{Probe Wait} state, node $N$ sends a \emph{Probe} message to each neighbour in turn. The neighbours respond to the message by sending eight packets to $N$ on the new channel $D$. 
The buffer can accommodate eight packets at a time. As the packets might not be sent immediately due to wakes up and collisions, sending more packets would have the risk of being dropped. 
The condition of the channel is further investigated through the number of retransmissions and packet collisions of the probing packets for accuracy of the channel condition. 

If the probing process times out (because of some communication failure) or the number of probe packets received is above a threshold (currently set to 16, including retransmissions and collisions) then node $N$ immediately exits \emph{Probe Wait} state and reverts to channel $C$ its previous channel. 


All neighbours are informed of the change back to channel $C$ and the LPBR is informed of the quality check failure with a summary of all probes received.
If, on the other hand, all channel quality checks succeed, the change to channel $D$ becomes permanent for node $N$ and it informs the LPBR of the results of the probing (numbers of packets received) and the channel change.

Probing is essential to make the channel change decision. It gives a quick overview of the channel condition based on the number of probing messages received. It is worth noting that probing is only done between the node and the tree neighbours. Neighbours that are not tree neighbours will not use the node as a route during their transmission thus, there is no need for probing to take place with those neighbours. However, the neighbours still need to know the channel value given that RPL control messages are sent to neighbours directly without using the routes.

\section{Reconnection Strategy}
RPL topology stability (using routing metric) remains the same in multi channel \cite{routingmetrics, winter2012rpl}.
%RPL topology could change according to the routing metric \cite{routingmetrics} the way the usual RPL would work \cite{winter2012rpl}. 
The nodes can still change the parents as usual as all neighbours know each other new channels. The neighbours that are not part of the route do not probe the parent when making the channel decision. However, the neighbours are informed of any channel changes.
This enables the topology to be optimised when communication fails and further improved through MCRP as the nodes have knowledge of the listening channels of all other nodes within the range. If a new node tries to join the topology, it sends a RPL control message through all channels as the listening nodes are unlikely to be on the default channel. The listening nodes send a broadcast on a default channel to discover new nodes (in Contiki default, new nodes will start on channel 26) and send RPL messages through unicast when the neighbours are known to reduce unnecessary transmissions in broadcast. New nodes and nodes which fall off the network can now rejoin on many potential channels.