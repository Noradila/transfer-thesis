\maketitle
\makedeclaration

\begin{abstract} % 300 word limit
%My research is about stuff.
%It begins with a study of some stuff, and then some other stuff and things.
%There is a 300-word limit on your abstract.

This report presents the current state of research work that has been carried out in the context of the PhD. (describe why do it). The PhD work proposes a new decentralised multi-channel tree building protocol with a centralised controller for ad-hoc sensor networks. The protocol alleviates the effect of interference which results in improved network efficiency and stability, and link reliability. The proposed protocol takes into account all available channels to utilise the spectrum and aims to use the spectrum efficiently by transmitting on several channels. The protocol detects which channels suffer interference and changes away from those channels. The algorithm for channel selection is a two-hop colouring protocol that reduces the chances of nearby nodes to transmit on the same channel. All nodes are battery operated except for the low power border router (LPBR). This enables a centralised channel switching process at the LPBR. The protocol is built based on the routing protocol for low power and lossy networks (RPL). In its initial phase, the protocol uses RPL's standard topology formation to create an initial working topology and then seeks to improve this topology by switching channels. The report discusses the main engineering and research challenges raised by the protocol, and describes and explains the principles and mechanisms used to support the proposed protocol. It then presents an extensive evaluation of the protocol and other other approaches. The implementation and evaluation of the protocol is performed using the Contiki framework. The report then describes the future main research issues that will be investigated in the context of this PhD.



In this report,

The proposed approach
The report discusses 
\end{abstract}

\begin{acknowledgements}
Acknowledge all the things!
\end{acknowledgements}

\setcounter{tocdepth}{2} 
% Setting this higher means you get contents entries for
%  more minor section headers.

\tableofcontents
\listoffigures
\listoftables

