\chapter{Energy vs Loss Tradeoff}
\label{energyLoss}

////citeSOFTWARE-BASED DUNKELS. implement an on-line energy estimation technique in Contiki. These approaches measure the state of components, such as radio, sensors and LEDS, to calculate consumption./////CHECK!

-requires more energy (to do MCRP) but reduce retransmissions in the long term
-how much energy than usual?
-improvement in loss when using MCRP?

Three are three main ways that have been exploited for an energy-efficient WSNs which are through MAC protocols, transmission range and routing protocols.

Energy consumption of a node depends on the interference patterns. Developed a generic method for capturing interference characteristics and predicting a node's energy consumption. (energy consumption prediction based on interference measurement). WSN radios use nearly the same energy in all active modes of operation (send, receive, listen). Capture an interference pattern at a deployment site and use this pattern to estimate the energy consumption of a node deployed at this location in the future. \cite{alexlifetime}

\cite{energyrpl} is an objective function for RPL that used node remaining energy as metric in the parent selection process. Energy-based OF characteristics in terms of node battery level estimation, path cost and node rank computation. A node selects the neighbor that advertises the greatest path cost value as parent. Compute the path cost (from node i to the sink) as the minimum node energy level (between parent path cost and its own energy). Uses the node's remaining energy as the main routing metric. The implementation makes use of a well-known battery theoretical model which they estimate at runtime the node battery lifetime for routing. Energy aware routing aims to use nodes with higher remaining power level. The network should be reorganized to find more interesting nodes for routing thereby a balancing on all nodes battery level should occur. Use the rank notion to avoid routing loops. Rank, record its relative position to other nodes with regard to DODAG root. 