\chapter{Energy vs Loss Tradeoff}
\label{energyLoss}
There have been many works done to study and estimate the nodes energy consumption in real time in order to prolong the network lifetime. There are three main ways that have been exploited for an energy-efficient WSN which are through MAC protocols and routing protocols. However, most solutions use other metrics to represent the energy consumption such as the radio duty cycle and end to end delay as it is more complex to compute the energy level from the battery-powered sensors. 

\section{Existing Energy Consumption}
Energy consumption estimation is often used for comparison between different nodes thus the voltage is not required to be computed. It is possible to measure the battery level for battery-powered sensors, however, it cannot be directly translated for energy estimation because of the non-linearity of batteries.

\subsection{Energy-based RPL}
RPL is a routing protocol that builds the topology based on the routing metrics and constraints for path calculation that are defined separately from the topology. This separation allows new metrics and constraints to be defined to fulfil the specific application and network optimisation criteria. A routing metric is used to evaluate the path cost. The routing metrics can be categorised into link and node metrics \cite{routingmetrics}. In node metrics, it can be the node state which provides information about the node characteristics, energy such as selecting nodes with higher residual energy or hop count. In link metrics, it includes the link throughput, latency or link reliability such as ETX. RPL lists the metrics that could be used. However, the implementation is left to the application. 

\cite{energyrpl} designed an OF for RPL that uses the node remaining energy as the metric during the parent selection of the topology. It aims to select nodes with higher remaining power level as the path for transmissions. The implementation uses a battery theoretical model \cite{sensornets13} to estimate the node's battery lifetime at runtime. The OF concentrates on the node battery level estimation, path cost and node rank computation in selecting a parent. The node that advertises the maximum greatest path cost is selected as the parent. The maximum path cost from the node to the sink is computed as the minimum node energy level. 

\subsection{Real Time Energy Estimation}
Powertrace system \cite{dunkels2011powertrace} is able to profile the power behaviour at the network layer for WSNs. It computes the energy consumption estimation of sensors at run time that depends on the software based on-line energy estimation mechanism \cite{dunkels2007software} and able to show per-activity power cost such as the different states for wake ups, transmissions and receptions of a node.
The software based on-line energy estimation mechanism is used to estimate the node's current energy consumption in real time. The on-line energy estimation is implemented in Contiki OS. The energy estimation module uses time measurements that can be directly obtained from the microprocessor on-chip timer. When the component is switched on to produce a time stamp. The time difference from when the component was on and when it later is switched off is computed. The current draw of the component is used to compute the total energy consumption estimation.

%Energy consumption estimation is often used for comparison between different nodes thus the voltage is not required to be computed. It is possible to measure the battery level for battery-powered sensors, however, it cannot be directly translated for energy estimation because of the non-linearity of batteries.

\cite{alexlifetime} developed a generic method to predict a node's energy consumption by capturing the interference patterns. The interference characteristics of a specific deployment site are captured to enable estimation of the node energy consumption when nodes are deployed at the same location in the future.

These energy consumption estimation solutions can be used to improve the network by using the information to reconstruct the topology.


\section{Energy Efficient MCRP}
An ongoing work in MCRP is in deciding a routing that considers the energy.
MCRP is a centralised channel switching process at the LPBR. As LPBR is fully powered and has unlimited memory, LPBR will compute the nodes energy level. The topology tree is routed at LPBR, thus LPBR is in charge to reconfigure the tree depending on the energy level that it computed. By using the energy estimation model, the nodes can send the measurements to LPBR. This (ringankan) burden load from the nodes to LPBR. Based on these information, the LPBR can compute the energy level and the LPBR will know all nodes energy level. Based on the energy level of each node and cost of the link (link is weighted based on ???) LPBR can sends a reconfigure message so that the nodes will route using the nodes that have a higher energy level.

\begin{equation}
\sum_{l \in links} = (c_{l} + c'_{l})d_{l}
\end{equation}

Consider the energy as the sum of all links used by a node; where it depends on the cost per message of the link l and the total descendent that can be reached via link l where it is assumed that the number of sent and received packets are the same for each descendent. However, the cost of the links upwards (towards LPBR) are different to the cost of the links downwards as upwards and downwards links use different channels. 

more descendent, more energy. Use node with higher battery level.
Or try to balance the topology?


%%(energy consumption prediction based on interference measurement). 
%%WSN radios use nearly the same energy in all active modes of operation (send, receive, listen). 

%%	Node Energy Object - It may sometimes be desirable to avoid selecting a node with low residual energy as a router. The routing protocol engine may compute a longer path for some traffic in order to increase the network life duration. It is not uncommon for self-supporting nodes to have a combination of primary storage, energy scavenging and secondary storage, leading to three different values for acceptable average current depending on the time frame being considered. Raw power and energy values are meaningless without the knowledge of the energy cost of sending and receiving packets, and lifetime estimates have no value without some higher-level constraint on the lifetime required of a device. For battery-powered devices, E_E is the current expected lifetime divided by the desired minimum lifetime, in units of percent. Node Energy (NE) object is used to provide information related to node energy and may be used as a metric or as constraint. 

%%-requires more energy (to do MCRP) but reduce retransmissions in the long term
%%-how much energy than usual?
%%-improvement in loss when using MCRP?

%%Three are three main ways that have been exploited for an energy-efficient WSNs which are through MAC protocols, transmission range and routing protocols.

%%Energy aware routing aims to use nodes with higher remaining power level. The network should be reorganized to find more interesting nodes for routing thereby a balancing on all nodes battery level should occur. Use the rank notion to avoid routing loops. Rank, record its relative position to other nodes with regard to DODAG root. 